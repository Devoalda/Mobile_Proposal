\chapter{Project Proposal}\label{project_proposal}

\section{Problem Statement}\label{problem_statement}
The challenge at hand involves new students at SIT facing a \textbf{significant academic leap} from their previous educational experiences. The transition to university life brings forth various obstacles, including \textit{adapting to a different teaching style} and \textit{managing a more demanding workload}. This situation is a common struggle among \textbf{freshmen}, impacting their overall learning experience. The goal is to understand and address these challenges effectively by designing supportive educational systems and strategies that aid students in \textit{navigating the complexities of university life}, fostering a \textit{smoother transition}, and enhancing their learning journey.

\section{Implications}\label{implications}
Without proactive measures to support students in their transition to university life, there is a risk of \textbf{increased academic stress}, \textit{reduced satisfaction with the learning experience}, and potentially \textit{higher dropout rates}. Recognizing and addressing these challenges is crucial for the overall success and well-being of students, emphasizing the need for institutions to invest in \textit{tailored support systems}, \textit{mentorship programs}, and \textit{teaching methodologies} that facilitate a more seamless adaptation to the demands of higher education.
 
\section{Persona}\label{persona}

\begin{center}
\begin{longtable}{|l|p{10cm}|}
\hline
\textbf{Name} & Alex \\
\hline
\textbf{Age} & 19 \\
\hline
\textbf{Gender} & Male \\
\hline
\textbf{Program} & Computing Science \\
\hline
\textbf{Location} & Singapore Institute of Technology (SIT), Singapore\\
\hline
\textbf{Background} & Alex is the first in his family to step into the realm of higher education. He is enthusiastic about the opportunities that university life offers but is equally apprehensive about the academic hurdles that may come his way. \\
\hline
\textbf{Goals} & Alex aspires to attain a quality education that opens doors to promising career prospects. Additionally, he seeks to establish new connections, make friends, and engage in extracurricular activities for a well-rounded university experience. \\
\hline
\textbf{Pain points} & Alex is concerned about the rigorous academic workload in college, unsure of how to strike a balance between studies and social life. The prospect of making friends and fitting into the university environment adds to his worries. \\
\hline
\textbf{Other important information} & A diligent worker, Alex is determined to succeed in his academic journey. While slightly reserved, he is keen on overcoming shyness to build meaningful connections with his peers. \\
\hline
\end{longtable}
\end{center}


\section{Proposed Solution}
\label{proposed_solution}

Building upon the identified challenges, we propose a multifaceted solution designed to elevate the university experience.

\subsection{A Peer-to-Peer Learning Platform}

Inspired by popular platforms like Reddit and Instagram, our application fosters a forum-like environment where students, like Alex, can:

\begin{itemize}
  \item Ask questions and seek answers directly from peers.
  \item Upvote and prioritize valuable responses through a refined voting system.
  \item Browse a curated feed of questions tailored to their interests.
  \item Follow specific tags and explore a wider range of inquiries through a dedicated "Explore" page.
  \item Manage their profiles and interactions effectively.
\end{itemize}

\subsection{Technology Stack}

This solution leverages a blend of technologies to ensure optimal functionality:

\begin{itemize}
  \item \textbf{Networking:} Robust interactions and information exchange are facilitated through APIs from Firebase and Google AI.
  \item \textbf{Databases:} Secure and efficient data storage and retrieval are achieved using a combination of SQLite and Firebase databases.
  \item \textbf{Multimedia:} The platform integrates video and photo functionalities, allowing students to capture and share memorable moments, enriching their learning experience.
  \item \textbf{Optical Character Recognition (OCR):} Streamlining academic endeavors, OCR technology converts printed or handwritten text into digital formats, enhancing organization and accessibility of study materials.
\end{itemize}

\subsection{Motivation}

This project is driven by a commitment to easing the academic transition for new university students. Recognizing the common challenges faced by freshmen, we aim to create a supportive online community where students can learn and grow together. This platform empowers introverted students to voice their questions and fosters a collaborative environment where peers can assist each other during this crucial phase.





